\documentclass{report}

\usepackage{amsmath}
\usepackage{syntax}
\usepackage{lib/bcprules} 

\author{Arthur Giesel Vedana}
\title{L1 Syntax and Documentation}
\begin{document}

\maketitle
\pagenumbering{gobble}
\newpage

\pagenumbering{roman}
\tableofcontents
\newpage

\addcontentsline{toc}{chapter}{Introduction}
\section*{Introduction}
The $L1$ programming language is a functional language.
It has a simple I/O system supporting only direct string operations. 
It is strongly and statically typed language supporting both explicit and implicit typing.

This document both specifies the $L1$ language and shows its implementation in F\#.
It is divided into 4 categories:
\begin{enumerate}
	\item Abstract Syntax and Semantics
	
		This defines the abstract syntax and semantics for the functional language.
		It only contains the bare minimum for the language to function, without any syntactic sugar. 
	\item Concrete Syntax
	
		This is the actual syntax when programming for $L1$.
		This defines all operators, syntactic sugar and other aspects of the language.
	\item Implementation
	
		Technical aspects on how $L1$ is implemented in F\#, showing the interpreter, evaluator and type inference.
	\item Change log
	
		A chronological list of changes made both to the language definition and its implementation.
\end{enumerate}

\newpage
\pagenumbering{arabic}
\part{Abstract Syntax and Semantics}


\end{document}
